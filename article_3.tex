

%----------------------------------------------------------------------------------------
%	PACKAGES AND OTHER DOCUMENT CONFIGURATIONS
%----------------------------------------------------------------------------------------

\documentclass[fleqn,10pt]{SelfArx} % Document font size and equations flushed left

\usepackage[english]{babel} % Specify a different language here - english by default

\usepackage{lipsum} % Required to insert dummy text. To be removed otherwised
\usepackage[utf8]{inputenc}
 
\setlength{\parindent}{1em}
\setlength{\parskip}{0.5em}
\renewcommand{\baselinestretch}{1.0}



%----------------------------------------------------------------------------------------
%	COLUMNS
%----------------------------------------------------------------------------------------

\setlength{\columnsep}{0.55cm} % Distance between the two columns of text
\setlength{\fboxrule}{0.75pt} % Width of the border around the abstract

%----------------------------------------------------------------------------------------
%	COLORS
%----------------------------------------------------------------------------------------

\definecolor{color1}{RGB}{0,0,90} % Color of the article title and sections
\definecolor{color2}{RGB}{0,20,20} % Color of the boxes behind the abstract and headings

%----------------------------------------------------------------------------------------
%	HYPERLINKS
%----------------------------------------------------------------------------------------

\usepackage{hyperref} % Required for hyperlinks
\hypersetup{hidelinks,colorlinks,breaklinks=true,urlcolor=color2,citecolor=color1,linkcolor=color1,bookmarksopen=false,pdftitle={Title},pdfauthor={Author}}

%----------------------------------------------------------------------------------------
%	ARTICLE INFORMATION
%----------------------------------------------------------------------------------------

\JournalInfo{2017} % Journal information
\Archive{Zheng Yang, getAbstract} % Additional notes (e.g. copyright, DOI, review/research article)

\PaperTitle{How to Cut Your Email Time in Half} % Article title

\Authors{Stephanie Vozza}

\Keywords{TIme Development --- Career Development --- Habits --- Self Development} % 
\newcommand{\keywordname}{Keywords} % Defines the keywords heading name

%----------------------------------------------------------------------------------------
%	ABSTRACT
%----------------------------------------------------------------------------------------
\AboutAuthor { \textbf{Stephanie Vozza} wrote a book The Five-Minute Mom's Club: 105 Tips to Make a Mom's Life Easier and the founder of TheOrganizedParent.com, she writes about business, productivity, and cool people. }

\Teaser{Getting back from addictively checking emails to boost productivity. }

%----------------------------------------------------------------------------------------

\begin{document}

\flushbottom % Makes all text pages the same height

\maketitle % Print the title and abstract box

%\tableofcontents % Print the contents section

\thispagestyle{empty} % Removes page numbering from the first page

%----------------------------------------------------------------------------------------
%	ARTICLE CONTENTS
%----------------------------------------------------------------------------------------

\section*{Title Information} 

\addcontentsline{toc}{section}{111} % Adds this section to the table of contents

\begin{itemize}[noitemsep] 
\item Title: How to cut your email time in half
\item Subtitle:
\item Author(s): Stephanie Vozza
\item Publisher: Fast Company Magazine
\item Year: 2017
\item Pages: 1
\end{itemize}


%------------------------------------------------

\section*{Ratings}
\begin{itemize}[noitemsep] % [noitemsep] removes whitespace between the items for a compact look
\item Applicability/ Importance: 9
\item Innovation: 6
\item Style: 8 
 
\end{itemize}

\section*{Focus}
\begin{description}
\item [Career \& Self-Development ]Time Management
\end{description}

\section*{Email}
Would you like to have more spare time with family or think daily emails are too many to handle. In fact, discovering an effective way to better manage the relationship between emails, saving hours from reading them, is a relevant public need. Stephanie Vozza, author of The Five-Minute Mom's Club: 105 Tips to Make a Mom's Life Easier, tells you the tips how to cut the email time half. The tips are practical, digested from the successful and high efficient people. Applying these tips, breaking the bad habits, will help you stay away from the email distractions and get more work done.


\section*{Take-aways}
\begin{itemize}
\item Constantly checking email is an addiction destroying your productivity, need to break. 
\item Email problem is basically people's own fault.
\item Put the time on important goals or relationships is better than emails. 
\item 5 effective tips guide you spend time on emails smarter. 
\end{itemize}

\section*{Recommendations}
Reading emails takes notable time of people's daily life. Discovering an effective way to better manage the relationship between emails, saving hours from reading them, is a relevant public need. Stephanie Vozza, author of The Five-Minute Mom's Club: 105 Tips to Make a Mom's Life Easier, tells you the tips how to cut the email time half. The tips are practical, digested from the successful and high efficient people. Applying these tips, breaking the bad habits, will help you stay away from the email distractions and get more work done. getAbstract recommends this report to individuals who want to save their time from emails. 

\section*{What you will learn}
In this summary, you will learn:
\begin{enumerate}
\item Why you should reduce the time spending on emails 
\item How to control your time dealing with emails. 
\end{enumerate} 

\section*{Summary}
\begin{flushleft}
People love or hate emails. In Adobe, almost half of its employees wish they could have less emails . Another persuasive study shows 70\% of people check work email after 6 p.m. Checking email is like an addiction, bringing pleasures. An additional reason, people fear of missing out. 
\par Since reading emails takes up a enormous time in daily life, the need of better controlling the email habit and largely reducing the amount time spend on it becomes relevant, the following 5 tips Stephanie Vozza gives explains how: 
\end{flushleft}

\begin{enumerate}

\item Put it on the schedule. 
Planning processing emails, treat as tasks,  scheduling on a calendar. Dealing emails once or 3 times a day is practical. 

\item Quit the CC
CC'ing emails to so many people brings back more. Meaning you will spend more time to read. A recommend way before sending out emails is to think twice. 

\item Set up rules
Filtering emails into separate categories using specified rules, you could use the method to determine how much time you'll spend on different messages.     

\item Stop using your inbox as a to-do list
Avoid treating email inbox as a to-do list, reading all incoming emails dramatically decreases productivity. Using "4 Ds" method, the mail is deletable? if not, delegatable? still not, could be done processing within 5 minutes? No, defers. 

\item Inform others
Let others know your intention help reduce the time spend on email. The daily allocated time for processing email shut off the related notifications.
\end{enumerate}

\section*{Key Quotes}
\begin{description}
\item "People have to understand that the email problem is largely their own fault."

\item "Just like sex, drugs, and many of us check email constantly because of the fear of missing out."

\item "Time spent on email is time that can be better spent on important goals or relationships." 
\end{description}

%----------------------------------------------------------------------------------------

\end{document}